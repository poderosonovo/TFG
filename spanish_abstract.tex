\begin{center}
\Large{Resumen}
\end{center}


\bigskip

Este \gls{TFG} se ha realizado en el entorno de la Cátedra BQ. BQ es una empresa que tiene en el desarrollo de teléfonos inteligentes, con el sistema operativo Android, una de sus principales actividades. 

El uso del sistema operativo Android en los dispositivos móviles es muy habitual, situándose en una cuota de mercado del 87\%. Debido a este crecimiento, su complejidad también ha ido creciendo.

En este marco se ha llevado a cabo la realización de un proyecto dedicado al análisis y procesado del audio sobre las versiones de Android 6.0 y 7.1.1. El objetivo principal es el diseño de nuevas estrategias que optimicen la gestión del audio en dicho sistema operativo.

En este proyecto se han llegado a abordar tres puntos de interés del tratamiento del audio a lo largo del sistema.

Por un lado, se ha estudiado los modos de clasificación del audio, realizando la prueba de implementar un nuevo modo de clasificación. 

Por otro lado, se ha analizado y configurado las diferentes curvas de volumen, encargadas del control del volumen en los distintos casos de uso. En este punto es en el que ha tenido lugar el mayor cambio tras la actualización de la versión 6.0 a 7.1.1. Por tanto, tras implementar dicho cambio, se han comparado las implementaciones de ambas versiones.

El último desarrollo ha consistido en el estudio del encaminamiento y procesado del audio en las capas inferiores del sistema operativo, las más cercanas al \textit{hardware}. Tras este estudio, ha sido posible diseñar y aplicar nuevas rutas de procesado, con el objetivo de solventar algunos de los problemas encontrados durante la realización de este proyecto.


\bigskip
\bigskip
\bigskip
\bigskip
\bigskip

\textbf{Palabras clave:} Audio, Android, Modos de audio, Curvas de volumen, Procesador digital de señales