\selectlanguage{english}
\begin{center}
\Large{Summary}
\end{center}

\bigskip

This \gls{TFG} has been achieved in Cátedra BQ surroundings. BQ is a company which works developing smartphones, running Android operating system, as one of its main products.

The use of Android as the operating system in smartphones is very usual, and its market share is around an 87\%. Due to this growth, its complexity has rised as well.

In this surrounding it has been developed a project dedicated to the analysis and process of audio in Android versions 6.0 and 7.1.1. The main objective is to design new strategies which optimize the audio control in this operating system.

Three interesting points about audio treatment along the system have been tackled in this project.

On one hand, audio modes classification have been studied, making the test of introducing a new audio mode.

On the other hand, volume curves in charge of volume control in different use cases have been analized and configured. Here is where the main change, after updating Android from 6.0 to 7.1.1, has taken place. Because of this, after implementing this change, a comparison between both implementations has been made.

The last development has consisted of a study about routing and processing of audio in lower layers of the Android operating system, which are the nearest layers to hardware. After this study, it has been possible to design new processing routes, and to apply them to resolve some problems found during project realization.


\bigskip
\bigskip
\bigskip
\bigskip
\bigskip

\textbf{Keywords:} Audio, Android, Audio modes, Volume curves, Digital signal processor
\selectlanguage{spanish}