\documentclass[a4paper, 12pt, twoside]{report}

\usepackage{parskip} %separación entre lineas
\usepackage[acronyms]{glossaries}
\usepackage[utf8]{inputenc}
\usepackage[spanish, es-tabla, es-nodecimaldot, english]{babel}
\usepackage{soul}%para tachar palabras
\usepackage{natbib} %para citar bibliografía 
\usepackage[pdftex, colorlinks=true, pdfstartview=FitV, linkcolor=black, citecolor=black, urlcolor=black, plainpages=false]{hyperref}
\usepackage{amsmath}%operaciones matemáticas
\usepackage{enumitem}%enumeraciones
\usepackage{fancyhdr}%para encabezados y pie de pagina
\usepackage{graphicx}%para más opciones al meter imágenes
\graphicspath{ {graphics/} }%Ruta de las imágenes
%Márgenes
\usepackage[left=1.2in, right=1.2in, top=1in, bottom=1.5in]{geometry}
% This allows setting a background colour for a page.
\usepackage{afterpage}
\usepackage{xcolor}

\newcommand{\BlankPage}{\newpage\null\thispagestyle{empty}\newpage}%Si ponemos \BlankPage añade una página en blanco
\addto\captionsspanish{\renewcommand{\contentsname}{Índice}} %Para que la tableofcontents se llama Índice en lugar de Contents

\begin{document}
\pagestyle{plain}
\pagenumbering{gobble}
\selectlanguage{spanish}
%%%%%%%%%%%%%%%%%
%TITLE
%%%%%%%%%%%%%%%%%
\input{title_page}
\BlankPage
\newenvironment{xdesc}[2]
  {\begin{description}[leftmargin=#2,labelindent=#1,labelwidth=#1, labelsep=0pt,align=left,style=multiline]}
  {\end{description}}

\begin{center}
	\large {\textsf{Trabajo de Fin de Grado}}
\end{center}

\bigskip

\begin{xdesc}{-1cm}{4cm}
\large{
	\item[Título:] Diseño de estrategias para la gestión del audio en teléfonos inteligentes sobre el sistema operativo Android.
	\item[Autor:] Diego Poderoso Novo
	\item[Tutor:] Alvaro Araujo Pinto
	\item[Departamento:] Ingeniería Electrónica
}
\end{xdesc}

\bigskip
\bigskip
\bigskip
\bigskip
\bigskip
\bigskip

\begin{center}
	\large{\textsf{Tribunal}}
\end{center}

\bigskip

\begin{xdesc}{-1cm}{4cm}
\large{
	\item[Presidente:] 
	\item[Vocal:] 
	\item[Secretario:] 
	\item[Suplente:]
	\item % se pone esta línea para que Suplente: aparezca
}
\end{xdesc}

\bigskip
\bigskip
\bigskip
\bigskip
\bigskip
\bigskip
\bigskip
\bigskip
\bigskip

\begin{flushleft}
\large{\textsf{Calificación:}}
\end{flushleft}

\bigskip
\bigskip

\begin{flushleft}
	\large{\textsf{Fecha de lectura:}}
	Madrid, a \hspace{1cm} de Junio de 2017
\end{flushleft}
\BlankPage
\begin{titlepage}
	\begin{center}
	
		\large {\textsc{\textsf{Universidad Politécnica de Madrid}}}
		
		\bigskip
		\bigskip
		
		\large {\textsf{Escuela Técnica Superior de Ingenieros de Telecomunicación}}
	
		\bigskip
		\includegraphics[height=3cm,keepaspectratio]{BlackOverWhite.png}
		\includegraphics[height=3cm,keepaspectratio]{die_logo.png}
		\includegraphics[height=3cm,keepaspectratio]{bq_logo.png}
		\bigskip

		\includegraphics[width=10cm,keepaspectratio]{logo_escuela.pdf}
		
		\bigskip
		\bigskip
		\bigskip
		
		\large {\textsc{\textsf{Trabajo de Fin de Grado}}}
		
		\bigskip
		\bigskip
		\bigskip
		
		\large {\textsf{Grado en Ingeniería de Tecnologías y Servicios de Telecomunicación}}
		
		\bigskip
		\bigskip
		\bigskip
		
		\large {\textsc{\textsf{Diseño de estrategias para la gestión del audio en teléfonos inteligentes sobre el sistema operativo Android}}}
		
		\bigskip
		
		\selectlanguage{english}
		\large {\textsc{\textsf{Design of strategies for audio management in smartphones over Android operating system}}}
		\selectlanguage{spanish}
		
		\bigskip
		\bigskip
		\bigskip
		\bigskip
		
		\large {Diego Poderoso Novo}
		\linebreak
		\large {Junio 2017}
				
	\end{center}
\end{titlepage}
\BlankPage
\begin{center}
\Large{Resumen}
\end{center}

\bigskip

%Añadir el resumen en español

\bigskip
\bigskip
\bigskip

\textbf{Palabras clave:} %añadir palabras clave en español
\BlankPage
\selectlanguage{english}
\begin{center}
\Large{Summary}
\end{center}

\bigskip

This \gls{TFG} has been achieved in Cátedra BQ surroundings. BQ is a company which works developing smartphones, running Android operating system, as one of its main products.

The use of Android as the operating system in smartphones is very usual, and its market share is around an 87\%. Due to this growth, its complexity has rised as well.

In this surrounding it has been developed a project dedicated to the analysis and process of audio in Android versions 6.0 and 7.1.1. The main objective is to design new strategies which optimize the audio control in this operating system.

Three interesting points about audio treatment along the system have been tackled in this project.

On one hand, audio modes classification have been studied, making the test of introducing a new audio mode.

On the other hand, volume curves in charge of volume control in different use cases have been analized and configured. Here is where the main change, after updating Android from 6.0 to 7.1.1, has taken place. Because of this, after implementing this change, a comparison between both implementations has been made.

The last development has consisted of a study about routing and processing of audio in lower layers of the Android operating system, which are the nearest layers to hardware. After this study, it has been possible to design new processing routes, and to apply them to resolve some problems found during project realization.


\bigskip
\bigskip
\bigskip
\bigskip
\bigskip

\textbf{Keywords:} Audio, Android, Audio modes, Volume curves, Digital signal processor
\selectlanguage{spanish}
\BlankPage



% Contents
\pagenumbering{roman}
\tableofcontents
\BlankPage
%\listoffigures
%\BlankPage
%\listoftables
%\BlankPage

%%%%%%%%%%%%%%%%%%%%%%%%%%%
% Body
%%%%%%%%%%%%%%%%%%%%%%%%%%%
\pagestyle{fancy}
\fancyhf{}
\fancyhead[le,ro]{\nouppercase{\leftmark}}
\fancyhead[ro]{\nouppercase{\rightmark}}
\fancyhead[re,lo]{\thepage}

\pagenumbering{arabic}
\setcounter{page}{1}

%Capítulo 1. Introducción
\end{document}