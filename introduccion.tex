\chapter{Introducción}
Este \gls{TFG} se lleva a cabo en la Cátedra BQ, que surge de la colaboración entre la empresa Mundo Reader S.L y la Escuela Técnica Superior de Ingenieros de Telecomunicación de la Universidad Politécnica de Madrid.

\section{Motivación}
La empresa BQ tiene entre sus productos principales una amplia gama de teléfonos inteligentes con sistema operativo Android. El sistema operativo Android se presentó en 2007 y desde entonces no ha hecho más que crecer, hoy en día se encuentra en el 87\% de los teléfonos inteligentes. Debido a este crecimiento su complejidad también ha ido creciendo, y su código es muy extenso. Como ejemplo de complejidad, el código con el que se ha trabajado, antes de ser compilado, pesa alrededor de 100\gls{GB}.

El audio en los teléfonos inteligentes ha cobrado más importancia, si cabe, que la que tenían en los teléfonos tradicionales. Esto se debe a que, hoy en día, se utilizan estos dispositivos no sólo para comunicarse, sino también como reproductores multimedia.

Unos de los principales problemas que aparecen en los terminales móviles es el manejo y gestión de audio y en este problema se encuadra el proyecto.

\section{Objetivos}
El objetivo principal es el diseño de nuevas estrategias de gestión del audio en los teléfonos móviles actuales, que trabajan sobre el sistema operativo Android en su versión 7.1.1, que permitan mejorar la robustez y calidad del sistema.

Este proyecto ha sido tutorizado por parte del departamento de audio, con el objetivo de que se estudien los diferentes puntos críticos que ellos no han abordado. De esta manera, se pretende ofrecer una visión más específica de las posibilidades del audio en este sistema operativo, para poder llevar a cabo nuevas implementaciones.

De esta manera, tras la consecución de estos objetivos, se pretende obtener un documento que sirva de referencia a la hora del tratamiento del audio en el sistema operativo Android.

\section{Acerca de este documento}
El \gls{TFG} se ha dividido en las distintas partes que han sido estudiadas durante la realización de este proyecto.

Se compone de los siguientes capítulos:
\begin{itemize}
	\item{\textbf{Análisis}: se aborda la evolución durante las distintas versiones que ha tenido el sistema operativo Android, centrándose más en profundidad en aquellos cambios relacionados con el audio. Además, se explica la situación actual del audio en Android, así como su estructura a lo largo del sistema.}
	\item{\textbf{Entorno de trabajo}: se introducen los recursos utilizados durante la realización del proyecto. Se explica como se encuentra dividido el código de Android y los diferentes programas de terceros que se han utilizado.}
	\item{\textbf{Modos de audio}: fue el primer caso de estudio del proyecto, los modos de audio. Se explica que son los modos de audio, la selección de ellos a nivel de código y la posibilidad de implementar nuevos modos.}
	\item{\textbf{Curvas de volumen}: se realiza un análisis del control del volumen en el sistema operativo, tanto desde el punto de vista de interfaz de usuario como desde el punto de vista de un desarrollador de bajo nivel. Además, se implementan las mejoras que trae el cambio de versión de Android 6.0 a 7.1.1.}
	\item{\textbf{Rutas del audio}: se trata el tema más importante del proyecto, como es el encaminado y procesado del audio a lo largo del sistema operativo. Una vez se comprende dicho encaminamiento, se procede al diseño de nuevas rutas que solventen algunos de los problemas que se han encontrado durante este proyecto.}
	\item{\textbf{Conclusiones}: se detallan las conclusiones generales a las que se ha llegado tras la realización del proyecto. Además, se proponen líneas futuras de desarrollo que serían interesantes una vez se ha estudiado la situación actual del audio.}
\end{itemize}