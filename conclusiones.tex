\chapter{Conclusiones y futuro desarrollo}
Antes de la realización de este \gls{TFG} se había fijado como objetivo principal el diseño de nuevas estrategias de gestión del audio en los teléfonos móviles, que trabajan sobre el sistema operativo Android, sin limitarse a ninguna de las capas de dicho sistema operativo.

Para poder realizar dicho objetivo, se habían fijado distintos subobjetivos:
\begin{itemize}
	\item{Análisis del funcionamiento del audio en las distintas capas del sistema operativo: se han analizado las clasificaciones del audio en las capas superiores, \gls{JNI} y \textit{framework} nativo, tal y como se ha visto en la sección \ref{sec:def_modos_audio}. Las capas intermedias, donde se realiza la implementación del volumen en Android, se han abordado en la sección \ref{sec:def_curvas}. Las capas inferiores, la \gls{HAL} y el \textit{kernel}, se han analizado para comprender la forma en que se ruta el audio en Android, como se ha visto en la sección \ref{sec:rutado_codigo}. Además, se ha analizado el rutado en el \gls{ADSP} y las topologías aplicadas en los distintos casos de uso en las secciones \ref{sec:rutado_DSP} y \ref{sec:conf_DSP}.}
	\item{Detección de problemas o posibles mejoras en cada una de las capas: en el capítulo \ref{chap:modos} se ha detectado que se interpreta de igual modo el tono de llamada de una llamada tradicional y de una \gls{VoIP}. En el capítulo \ref{chap:curvas_volumen} se ha observado que los cambios que trae la versión de Android N, con respecto al audio, no estaban implementados. En el capítulo \ref{chap:rutas_audio} se han observado problemas a la hora de aplicar procesados de supresión de ruido, también se han descubierto ciertas incompatibilidades en el funcionamiento de las llamadas \gls{VoIP}.}
	\item{Implementación de las soluciones o mejoras: para diferenciar entre el tono de llamada de una llamada tradicional y una \gls{VoIP} se ha implementado un nuevo modo de audio tal y como se describe en la sección \ref{sec:nuevo_modo}. Se han habilitado las mejoras que trae Android N, con respecto a las curvas de volumen, requiriendo el ajuste de las nuevas curvas de volumen tal y como se tenía en Android M. Esta implementación se ha llevado a cabo en la sección \ref{sec:dif_N_M}. La implementación de una nueva topología de procesado en el \gls{ADSP}, que resolviese los problemas del módulo de supresión de ruido, se ha realizado tal y como se indica en la sección \ref{sec:supresion_ruido}. Como última implementación, se han realizado nuevas rutas asociadas a las llamadas \gls{VoIP} para solventar algunos de los problemas encontrados en su uso, esta implementación se explica en la sección \ref{sec:llamadas_VoIP}.}
	\item{Realización de pruebas sobre las nuevas implementaciones: para cada una de las implementaciones anteriormente descritas se han realizado distintas pruebas. En la sección \ref{sec:pruebas_nuevo_modo} se detallan los problemas obtenidos tras la implementación de un nuevo modo de audio. En la sección \ref{sec:dif_N_M} no se ha observado ninguna incompatibilidad a la hora de aplicar la nueva implementación de curvas de volumen, manteniéndose, de cara al usuario, la misma funcionalidad. En la sección \ref{sec:supresion_ruido} se han realizado distintas pruebas del nivel de ruido en las grabaciones obtenidas con aplicaciones de mensajería y la aplicación nativa de grabación, comparando grabaciones realizadas cuando se tenía activa la nueva topología y cuando no lo estaba. En la sección \ref{sec:llamadas_VoIP} se han realizado llamadas \gls{VoIP} con las aplicaciones que tenían ciertas incompatibilidades, además se ha comprobado el uso de la captación de audio estéreo en el caso de poner el dispositivo en manos libres durante la llamada.}
	\item{Conclusiones sobre los resultados obtenidos: como conclusión de la implementación de un nuevo modo de audio se obtiene que no es conveniente su realización, pues se estarían modificando las \gls{API} sin advertir al desarrollador de aplicaciones. Con respecto a la nueva implementación de las curvas de volumen, permite modificaciones más rápidas, ya que no es necesario recompilar el código al cambiar los ficheros que ajustan las curvas. En cuanto a la nueva topología de supresión de ruido se han obtenido resultado muy satisfactorios, produciéndose una disminución considerable del ruido cuando se aplica. Por último, durante las llamadas \gls{VoIP} la experiencia ha sido la esperada, con un nivel de volumen adecuado y una estabilidad de la llamada correcta. Además, la captación estéreo del audio durante una llamada se aplica únicamente cuando se está en el modo manos libres, como se pretendía.}
\end{itemize}

Por tanto, se puede asegurar que se han cumplido todos los objetivo fijados tras la realización de este proyecto.

Como conclusiones generales se ha podido observar que las versiones del sistema operativo Android 6.0 y 7.1.1, que han sido con las que se ha trabajado, son versiones muy estables. Tal y como se ha visto en este proyecto, la mayoría de los problemas relacionados con el audio en Android provienen de  ciertas incompatibilidades por parte del \gls{SoC} a las peticiones del sistema, o por un uso incorrecto de las \gls{API} por parte de los desarrolladores de aplicaciones.

Como desarrollo futuro, sería interesante estudiar más a fondo el problema de la latencia del audio en Android, aún demasiado alta y variable según para que caso de uso. Además, se está trabajando para encontrar la relación entre el fichero mixer.xml, que se encuentra en la \gls{HAL} y se encarga de configurar las rutas del códec de audio, y la última capa del sistema, el \textit{kernel} de Android, donde se realizan las escrituras en los registros de memoria necesarios.



