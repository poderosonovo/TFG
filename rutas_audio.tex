\chapter{Rutas del audio}\label{chap:rutas_audio}
En este capítulo se explicará el rutado del audio tanto a nivel \textit{software}, como a nivel \textit{hardware}. Se abordarán temas como el procesado del audio, que es dependiente del camino que se escoja, como la configuración de dicho procesado y la asociación de nuevas rutas. Por tanto, se entrará en detalle a explicar como configurar el \gls{ADSP} y como evaluar los resultados de dichas configuraciones.


\section{Rutado en el código}\label{sec:rutado_codigo}
Para explicar todo esto, es necesario tener claras las diferencias que existen en Android sobre el rutado a nivel de código. Una parte del código se encargará de rutar el audio por el camino del \gls{ADSP}, y otra parte se encargará de rutarlo por los distintos módulos \textit{hardware}.

Para empezar, la selección de un camino u otro se realiza a nivel de \gls{HAL}. Esto implica que estamos a un nivel más bajo del que se había trabajado hasta ahora.

En el siguiente diagrama se pueden observar los diferentes bloques por los que se ruta el audio en función de lo que indiquen algunos ficheros de configuración:

\begin{figure}[H]
	\centering
	\includegraphics[width=0.8\textwidth]{rutas_esquema_audio}
	\caption{Esquema del rutado del audio a bajo nivel.}
	\label{figrutas_esquema}
\end{figure}

\begin{itemize}
	\item{platform.c: fichero del código que se encarga de la lógica de rutado. Para cada caso de uso asocia una topología, un camino del \gls{ADSP}, y un camino del codec de audio, que puede ser el WCD8x16 o el TFA9911.}
	\item{\gls{ADSP}: se encarga del procesado digital de la señal de audio. Sus topologías estarán configuradas según unos ficheros en el sistema, los ficheros \gls{ACDB}. No incluye lógica de rutado.}
	\item{WCD8x16: es el códec de audio de Qualcomm. Se encarga del rutado y procesado del audio al más bajo nivel. Su configuración se realiza mediante un fichero mixer.xml. Incluye, por tanto, lógica de rutado. Recibe el audio de los micrófonos, lo procesa y se lo envía al \gls{ADSP}. También, recibe el audio del \gls{ADSP} y se encarga de sacarlo por los auriculares, mediante el conector de audio analógico de 3.5mm.}
	\item{TFA9911: es un códec de audio dedicado exclusivamente para el altavoz del dispositivo. Incluye su propia lógica de rutado y su propio \gls{ADSP}. También se encarga del rutado y procesado al más bajo nivel. Se configura mediante un fichero tfa.cnt que configura diferentes perfiles de sonido para el altavoz.}
\end{itemize}

\subsection{Rutado a nivel de DSP} \label{sec:rutado_DSP}
Como ya se ha comentado, en la \gls{HAL} existe un fichero platform.c que es el encargado de la lógica de rutado del audio. Este fichero, en función de las características de la petición que reciba, habilitará un camino u otro.

En el fichero platform.h se declaran todos los tipos de dispositivos. Estos dispositivos llevarán asociados una ruta del \gls{ADSP} y una ruta del códec de audio. Por tanto, para un sonido por el mismo altavoz, pueden existir muchos dispositivos para aplicarse en diferentes casos de uso.

Se declaran dos tipos de dispositivos. Por un lado, se encuentran los dispositivos asociados con la salida del sonido, la reproducción. Por otro lado, se encuentran los dispositivos asociados con la entrada de sonido al sistema, la captación de audio. La declaración de estos dispositivos se realiza en el fichero platform.h de la siguiente manera:

\begin{figure}[H]
	\centering
	\subfigure[Dispositivos de reproducción.]{\includegraphics[width=76mm]{declaracion_out_snd}}
	\subfigure[Dispositivos de captación.]{\includegraphics[width=71mm]{declaracion_in_snd}}
	\caption{Declaración de dispositivos para cada ruta.} \label{fig:declaracion_snd}
\end{figure} 

La lista de dispositivos declarados es muy grande, existiendo más dispositivos que rutas. Normalmente, cada dispositivo que se declara sirve para un caso particular. Aunque, en la realidad, muchos de los dispositivos se asocian a diferentes situaciones, ya que no sería mantenible configurar tantas topologías del \gls{ADSP}, ni rutas en el códec de audio.

En el fichero platform.c es donde se asocia cada uno de estos dispositivos con una topología y un camino del códec de audio. La asociación a nivel de \gls{ADSP} se realiza mediante unos números que identifican cada una de las topologías. De este modo, para asociar los dispositivos anteriormente declarados se haría así:

\begin{figure}[H]
	\centering
	\subfigure[Dispositivos de reproducción.]{\includegraphics[width=76mm]{acdb_id_table_out}}
	\subfigure[Dispositivos de captación.]{\includegraphics[width=71mm]{acdb_id_table_in}}
	\caption{Asociación de topologías para cada dispositivo.} \label{fig:asociacion_acdb_id}
\end{figure} 

Pero hasta ahora estos dispositivos sólo se han declarado y configurado, falta seleccionarlos en cada caso de uso. Para ello, existen dos métodos de obtención del dispositivo, uno para salida y otro para entrada de audio. 

En estos métodos se hace uso de todas las características de las peticiones del audio. Estas pueden ser: los modos de audio ya comentados en el capítulo \ref{chap:modos}, los dispositivos por los que vaya a sonar o captar el audio, los tipos de flujos de audio, etc. En función de todo lo anterior y muchas otras características de las peticiones, se asocian los dispositivos.

El método encargado de la selección del dispositivo de salida se denomina platform\_get\_output\_snd\_device. El método encargado de la selección de los dispositivos de entrada se denomina platform\_get\_input\_snd\_device. Destacar que se trata de métodos muy largos, con unas doscientas líneas de código, y con una lógica muy enrevesada. Por tanto, no son métodos muy mantenibles.

Pero las topologías del \gls{ADSP} no dependen únicamente del dispositivo asociado, también dependen del tipo de aplicación. Esto es un identificador, que en función del tipo de petición, asociará una topología u otra, para un dispositivo seleccionado.

En el fichero platform.c se definen los tipos por defecto tanto para recepción, encargada de la reproducción de sonido, como para transmisión, encargada de la captación de sonido. Su declaración y asociación se puede observar a continuación:

\begin{figure}[H]
	\centering
	\includegraphics[width=0.5\textwidth]{declaracion_app_type}
	\caption{Tipos de aplicación por defecto.} 
	\label{fig:default_app_type}
\end{figure} 

La asociación de dichos tipos se realiza, también, en el fichero platform.c en función de si se trata de una petición de recepción o de transmisión. Además, para el caso de recepción, otros tipos de aplicación se pueden asociar. Esta asociación se realiza en el fichero de la \gls{HAL} audio\_output\_policy.conf, pero por lo general, los casos por defecto son los que se utilizan.

Cabe destacar que en el caso de estar tratando con una llamada de voz, y que el código la interprete correctamente, los tipos de aplicación no se utilizan. Sólo se utilizan en los casos de reproducción o de grabación de audio.

Teniendo todo esto en cuenta, el rutado hacia una topología del \gls{ADSP} se haría correctamente.

\subsection{Rutado a nivel de \textit{hardware}}
Tras pasar por el \gls{ADSP}, el audio tiene que ir por un camino físico hasta el dispositivo que se encargará de reproducirlo.

Este camino, como ya se ha comentado, se selecciona con el fichero mixer.xml, el cual contiene los diferentes caminos existentes en el \textit{hardware}.

La selección de uno de estos caminos también se realiza en el fichero platform.c. Se asocia a uno de los dispositivos, de manera similar a los identificadores del \gls{ACDB}. Al igual que en ese caso, hay que asociar tanto para dispositivos de salida como de entrada:

\begin{figure}[H]
	\centering
	\subfigure[Dispositivos de reproducción.]{\includegraphics[width=76mm]{snd_out_mixer}}
	\subfigure[Dispositivos de captación.]{\includegraphics[width=71mm]{snd_in_mixer}}
	\caption{Asociación de rutas \textit{hardware} para cada dispositivo.} \label{fig:asociacion_snd_mixer}
\end{figure}

Una vez seleccionado el tipo de dispositivo se aplicará la ruta definida en el fichero mixer.xml.

Este fichero se compone de tres partes:
\begin{itemize}
	\item{Configuración inicial: en esta zona se definen los valores por defecto que tendrán los diferentes elementos \textit{hardware}. Se definen los estados en reposo de: líneas, buses, amplificadores, multiplexores, filtros, conversores, etc.}
	\item{Configuraciones específicas: en esta zona se definen configuraciones más específicas en función de los requisitos de latencia, el uso del \gls{HDMI}, el uso de la radio \gls{FM}, el uso del Bluetooth, etc.}
	\item{Configuraciones de cada dispositivo: en esta zona es dónde se definen las distintas rutas para cada dispositivo. Muchos de ellos utilizan las mismas configuraciones.}
\end{itemize}

Un ejemplo de ruta para los dispositivos SND\_DEVICE\_OUT\_HANDSET y \\ SND\_DEVICE\_IN\_HANDSET\_MIC queda definido de la siguiente forma:

\begin{figure}[H]
	\centering
	\subfigure[Dispositivo de reproducción.]{\includegraphics[width=76mm]{mixer_handset}}
	\subfigure[Dispositivo de captación.]{\includegraphics[width=71mm]{mixer_handset_mic}}
	\caption{Definición de rutas \textit{hardware}.} \label{fig:definicion_rutas_mixer}
\end{figure}

Esto asocia a unas etiquetas ciertos valores. A más bajo nivel, al nivel del \textit{kernel}, estos valores se interpretan y se escribe en los registros necesarios.

En el caso del dispositivo de reproducción, que está asociado a la reproducción por el auricular del teléfono, se transmite el audio por el camino RX1, configurándose un volumen, una ganancia del amplificador y configurando que el sonido salga por el auricular, para lo cual se da a la etiqueta EAR\_S el valor Switch.

Esto nos permite abstraernos de la lógica a nivel de registros, con el uso de un lenguaje de programación y una topología más simple.

Se puede consultar el valor de cada etiqueta en tiempo real. Para ello, es necesario tener conectado el dispositivo al ordenador vía \gls{USB}. Se hace uso de las herramientas \gls{ADB} y del comando tinymix. Además, con este comando también sera posible la modificación en tiempo real de las etiquetas.

En la siguiente imagen se puede observar como se consulta el valor del amplificador del auricular y se modifica, produciéndose una bajada de volumen:

\begin{figure}[H]
	\centering
	\includegraphics[width=0.5\textwidth]{tinymix_ear_pa}
	\caption{Modificación de la ganancia del auricular en tiempo real.} 
	\label{fig:tinymix_ear_pa}
\end{figure}

El valor, cuando se está escuchando un audio por el auricular en el móvil, es de una ganancia de 6\gls{dB}, pero mediante el comando tinymix se puede forzar a que baje a 1.5\gls{dB}.

Como caso particular, falta por comentar el altavoz. Este altavoz tiene un códec de audio propio, por tanto para que se rute correctamente ha de añadirse, para todos los dispositivos que vayan a reproducirse por ahí, las siguientes líneas en el fichero audio\_hw.c:

\begin{figure}[H]
	\centering
	\includegraphics[width=1\textwidth]{mixer_smpa}
	\caption{Rutado por el códec de audio del altavoz.} 
	\label{fig:mixer_smpa}
\end{figure}

De esta manera, para cada dispositivo relacionado con el altavoz, se asocia una nueva etiqueta que hará que el audio se rute por el códec TFA9911 en lugar de por el WCD8x16.

\section{Configuración del DSP a través del programa QACT} \label{sec:conf_DSP}
La configuración del \gls{ADSP} únicamente es posible realizarla mediante el programa \gls{QACT}. Este programa sólo funciona en el sistema operativo Windows, por tanto para realizar esta implementación se trabajará con dicho sistema.

Este programa permite dos tipos de conexión:
\begin{itemize}
	\item{Modo sin conexión: permite editar los ficheros \gls{ACDB} sin tener conectado el dispositivo, siempre que los tengamos en el directorio local del ordenador. En este modo se permite, no sólo la configuración, sino también la creación y asociación de nuevas topologías. Tras editar los ficheros, será necesario cargarlos en el teléfono.}
	\item{Modo con conexión: permite realizar la configuración de los ficheros \gls{ACDB} que se encuentran en el dispositivo. También permite observar que topología se aplica en tiempo real y modificarla. Los cambios realizados en tiempo real no serán permanentes. Para poder utilizar este modo, es necesario estar conectado mediante el programa \gls{QPST} al ordenador.}
\end{itemize}

La página principal del programa es la siguiente:

\begin{figure}[H]
	\centering
	\includegraphics[width=1\textwidth]{QACT_MainPage}
	\caption{Página principal del programa QACT.} 
	\label{fig:qact_main}
\end{figure}

En ella se puede observar que hay tres desplegables que seleccionan el caso de uso que se muestra:
\begin{itemize}
	\item{Caso de uso del audio: puede ser grabación, reproducción o voz. Cada dispositivo debe ser de una de estas clases.}
	\item{Dispositivo: se selecciona entre un conjunto de dispositivos, que serán los que tengan asociados el identificador que se asocia en el rutado del código, como se muestra en la figura \ref{fig:asociacion_acdb_id}. Se pueden añadir, modificar y eliminar dispositivos.}
	\item{Frecuencia de muestreo: cada caso de uso puede ser configurado para cada una de las frecuencias de muestreo que soporta.}
\end{itemize}

Además, se puede observar que una ruta se divide en tres áreas, con distintas topologías en cada área. Estas áreas son:
\begin{itemize}
	\item{\gls{AFE}: sólo una topología de esta área puede ser asociada a cada dispositivo.}
	\item{\gls{COPP}: se asocia una topología de esta área para cada tipo de aplicación en cada dispositivo.}
	\item{\gls{POPP}: se asocia una topología distinta de esta área para cada tipo de aplicación, siendo común para todos los dispositivos. Es común para todos los dispositivos.}
\end{itemize}

Cada topología se compone de módulos ordenados. Estos módulos realizan distintos procesados y se pueden configurar seleccionándolos. Existen módulos de: filtrado, amplificación, supresión de ruido, compresión multibanda, muestreo, generación de ruido, potenciación de bajos, ecualización, etc.

\subsection{Modos de diseño disponibles}
Este programa permite diferentes modos de modificar el \gls{ADSP}, estos son: 
 
\begin{itemize}
	\item{Ajustar los módulos aplicados: se pueden configurar los módulos aplicados a cada topología, siempre que la topología este asociada a un caso de uso. Hay módulos que únicamente permiten ser activados o desactivados, pero otros, como los algunos filtros o compresores, permiten una configuración mucho más precisa.}
	\item{Reordenar los módulos de una topología aplicada: se pueden cambiar el orden en el que se aplican los módulos de procesado.}
	\item{Cambiar la topología aplicada: se pueden modificar la topología que se quiere aplicar en cada área. Habrá que tener en cuenta que si las cambiamos en la \gls{AFE} se cambiará para todo el tipo de dispositivo, y si la cambiamos en la \gls{POPP} será común para todos los tipos de dispositivos. Por esa razón, los cambios más habituales se suelen hacer en la \gls{COPP} en función del tipo de aplicación. Las topologías a aplicar en un área tienen que pertenecer a esa área en concreto.}
	\item{Crear una nueva topología: se puede crear desde cero una nueva topología. Se deben seleccionar los módulos de procesado que incluirá y el orden de estos. Sobre las topologías ya creadas no se pueden añadir nuevos módulos. Además, debe seleccionarse el área al que pertenecerá. Se debe tener en cuenta que, en las áreas \gls{COPP} y \gls{POPP}, se distingue entre topologías orientadas al uso en llamadas y entre el resto de topologías. Tras esto, podrá aplicarse al caso de uso que queramos.}
	\item{Crear un nuevo tipo de dispositivo: se puede crear desde cero un nuevo dispositivo, al que habrá que
	asociar un nuevo identificador tanto a nivel de \gls{ADSP} como a nivel de código. Se debe configurar lo siguiente:}
	\begin{itemize}
		\item{Categoría del dispositivo: si se aplicará dicho dispositivo a un caso de uso de Bluetooth, Auricular, \gls{HDMI}, Cascos, Altavoz o General.}
		\item{Identificador del dispositivo: es necesario definir el mismo identificador a nivel de \gls{ADSP} y a nivel de código, para que pueda rutarse por el nuevo dispositivo que se está creando.}
		\item{Frecuencias de muestreo soportadas por el dispositivo.}
		\item{Configuración del canal: hay que definir si el dispositivo se va a aplicar a un canal mono, estéreo o de un mayor número de canales.}
		\item{Dirección: indicar si el dispositivo se va a usar para recepción o para transmisión. Si el dispositivo se va a aplicar a un caso de uso de llamada, además, será necesario definirlo como un par nuevo de dispositivos junto con otro de recepción o transmisión, según sea el dispositivo que se ha diseñado. Para que, cuando se rute por él, se apliquen ambos caminos.}
		\item{El número de bits por muestra que el dispositivo soporta.}
		\item{Topologías utilizadas: se debe asociar una topología del área \gls{AFE} y una topología por cada tipo de aplicación, del área \gls{COPP}. Además, habrá que asociar una topología de voz del área \gls{COPP}, que no depende del tipo de aplicación.}
		\item{Retardos: para cada frecuencia de muestreo se puede indicar un retardo a aplicar.}
	\end{itemize}
	\item{Crear nuevos tipos de aplicación: se pueden crear nuevos tipos de aplicación y asociarles una topología de \gls{POPP}. Este nuevo tipo será necesario declararlo y asociarlo en el código, como se ha visto con los tipos de aplicación por defecto.}
\end{itemize}

Una vez se ha realizado alguna modificación de las indicadas anteriormente, en los ficheros \gls{ACDB} será necesario cargar estos ficheros en el sistema del teléfono. Para ello, deben obtenerse permisos de superusuario, habilitar la escritura del sistema, y sustituir los ficheros \gls{ACDB} existentes por los nuevos. Finalmente, para que se apliquen los cambios, es necesario reiniciar el dispositivo.

\section{Análisis de la salida de las áreas del DSP a través del los programas QXDM y QCAT} \label{sec:salida_qxdm}
En esta sección se explicará las posibilidades que ofrece el fabricante Qualcomm para depurar a nivel de \gls{ADSP} y códec de audio. Ambas capas, al ser propietarias de Qualcomm, se encuentran muy protegidas.Por tanto, una forma sencilla de obtener información es mediante su \textit{software}.

El programa \gls{QXDM} nos permite obtener trazas de los diferentes flujos de información que pasan por ciertos puntos. Para poder hacer uso de este programa, el dispositivo tiene que estar conectado al ordenador mediante el programa \gls{QPST}.

Este programa permite obtener trazas para muchas aplicaciones, para configurar que las que se quieren obtener son las trazas de audio es necesario en cuales puntos se pueden sacar. Se realiza una distinción entre las trazas obtenibles en una llamada, y las trazas obtenibles en el resto de casos de uso:

\begin{itemize}
		\item{Trazas en llamada: se pueden obtener en los puntos indicados posteriormente. El número hexadecimal es un identificador que será el que haya que configurar en el \gls{QXDM}.}
		\begin{itemize}
			\item{[0x1586]: trazas entre el área de \gls{AFE} y el códec de audio.}
			\item{[0x158A]: trazas a la entrada y salida del \gls{COPP} en transmisión, y sólo a la salida en recepción.}
			\item{[0x158B]: trazas a la entrada y salida del \gls{POPP}.}
			\item{[0x1804]: trazas entre el área de \gls{POPP} y los \gls{CVD} en transmisión.}
			\item{[0x1805]: trazas entre el área de \gls{POPP} y los \gls{CVD} en recepción.}
			\item{[0x14D2]: trazas del \gls{APR} del \gls{ADSP}.}
			\item{[0x14D0]: trazas del \gls{MVS} o \gls{VS}.}
			\item{[0x7143]: trazas del \gls{MVS} a la pila de protocolo del módem.}
			\item{[0x1009]: trazas del \gls{VS} a la pila de protocolo del módem.}
			\item{[0x7144]: trazas de la pila de protocolo del módem al \gls{MVS}.}
			\item{[0x100A]: trazas de la pila de protocolo del módem al \gls{VS}.}
		\end{itemize}
\end{itemize}

\begin{figure}[H]
		\centering
		\includegraphics[width=1.1\textwidth]{voice_QXDM}
		\caption{Puntos posibles donde obtener trazas en una llamada.} 
		\label{fig:voice_qxdm}
\end{figure}

\begin{itemize}
	\item{Trazas en el resto del audio: se pueden obtener en los siguientes puntos:}
	\begin{itemize}
		\item{[0x1586]: trazas entre el área de \gls{AFE} y el códec de audio.}
		\item{[0x1532]: trazas entre el área de \gls{AFE} y el área \gls{COPP} en transmisión.}
		\item{[0x1533]: trazas entre el área de \gls{COPP} y la matriz de transmisión.}
		\item{[0x1531]: trazas entre la matriz de recepción y el área \gls{COPP}.}
		\item{[0x1534]: trazas entre la matriz de transmisión y el área de \gls{POPP}.}
		\item{[0x1530]: trazas entre el área de \gls{POPP} y la matriz de recepción.}
		\item{[0x1535]: trazas entre el área de \gls{POPP} y el codificador de audio.}
		\item{[0x152F]: trazas entre el descodificador de audio y el área de \gls{POPP}.}
		\item{[0x1536]: trazas a la salida del codificador.}
		\item{[0x152E]: trazas a la entrada del descodificador.}
	\end{itemize}
\end{itemize}

\begin{figure}[H]
	\centering
	\includegraphics[width=1\textwidth]{audio_QXDM}
	\caption{Puntos posibles donde obtener trazas para el resto de audio.} 
	\label{fig:audio_qxdm}
\end{figure}

En función a estos puntos posibles, se deberá configurar en el programa \gls{QXDM} una vista filtrada que proporcione las trazas en las que se está interesado. Una vez se tenga esta vista, se debe reproducir el caso de uso que se quiera capturar.

Una vez obtenidas las trazas, estas se encontrarán en un formato propietario de Qualcomm. Para poder interpretarlas será necesario utilizar el programa \gls{QCAT}. Con este programa, se podrán obtener los flujos de audio, en formatos WAV o RAW, que se podrán analizar con cualquier programa de edición de audio. Se generarán al menos dos ficheros, uno por cada formato, por cada punto del que se han obtenido trazas. En los puntos comunes a transmisión y recepción, se diferenciará entre ambos caminos, creando dos ficheros distintos.

Además, este programa permite sustituir las trazas perdidas, rellenando en función del resto de trazas los huecos temporales que existan. De esta manera, se consigue una reconstrucción del audio completa en cada punto. 

Como ya se ha comentado en la sección \ref{sec:programas}, el programa de edición de audio que se ha utilizado para el análisis de los ficheros de audio obtenidos ha sido Adobe Audition.

\section{Diseño de nuevas rutas}
En este apartado se aplicará lo explicado en las secciones \ref{sec:rutado_codigo}, \ref{sec:conf_DSP} y \ref{sec:salida_qxdm} para abordar algunos problemas que han surgido durante la realización de este proyecto y que tienen que ver con la configuración y asociación de rutas del \gls{ADSP}. 

\subsection{Para la supresión de ruido en grabaciones} \label{sec:supresion_ruido}
El problema es que las rutas del \gls{ADSP} que se aplicaban en los casos de uso en que se iba a grabar un audio, ya sea con una aplicación de mensajería o con la propia aplicación de grabación del sistema, la supresión de ruido no estaba activa. Esto se debía a que, cuando lo estaba, tardaba en activarse unos segundos desde que se iniciaba la grabación. Por tanto, durante esos primeros segundos de grabación no se reducía el ruido ambiente, pero en el resto de grabación si que se reducía. Este efecto no era el deseado y por tanto se tenía desactivado el módulo encargado de ello.

Para solucionarlo se debe implementar una nueva topología que ya no tenga este problema. Esto se ha conseguido gracias al uso de una segunda versión del módulo de supresión de ruido el cual no produce dicho efecto, sino que permanece activo desde el principio.

Para identificar en cuales tipos de dispositivos habría que cambiar la topología aplicada, se ha observado el identificador del \gls{ACDB} asociado cuando se graba un audio. Las condiciones evaluadas han sido:

\begin{itemize}
	\item{Al grabar un audio con la aplicación de mensajería Whatsapp: el identificador asociado es el número cuatro. Por tanto, en la lista de tipos de dispositivos existente en el \gls{ADSP}, se busca el que tenga asociado el identificador número cuatro. En él, habrá que cambiar la topología asociada al tipo de aplicación por defecto de transmisión por la nueva topología creada.}
	\item{Al realizar una grabación mono con la aplicación Grabadora del sistema: el identificador asociado también es el número cuatro. Por tanto, el cambio anteriormente descrito es suficiente.}
	\item{Al realizar una grabación estéreo con la aplicación Grabadora del sistema: el identificador asociado es el número treinta y cuatro. De igual manera, se busca en la lista de dispositivos el que tenga asociado ese número. Se sustituye la topología asociada al tipo de aplicación por defecto de transmisión por una nueva topología. Esta topología ha de ser diferente, pues es grabación estéreo en lugar de mono, que es la que se había aplicado en los dos casos anteriores.}
	\item{Al realizar una grabación con cascos con cable que tengan micrófono incorporado: el identificador asociado es el número ocho. La topología que se pondrá será la misma que en los dos primeros casos, pues se trata de grabación mono.}
\end{itemize}

Por tanto, ha sido necesario crear dos topologías, ambas añadidas al área de \gls{COPP}. La nueva topología de grabación mono es la siguiente:

\begin{figure}[H]
	\centering
	\includegraphics[width=1\textwidth]{topologia_mono_eans2}
	\caption{Nueva topología de grabación mono con supresión de ruido.} 
	\label{fig:mono_eans2}
\end{figure}

Se pueden observar como se aplica primero un filtrado paso alto. Posteriormente, se amplifica el sonido con una ganancia de 18\gls{dB}. El siguiente módulo activo es la segunda versión del que suprime el ruido. El siguiente módulo activo es un compresor multibanda. Finalmente, se aplica un módulo que vuelve a muestrear a la salida del área.

La configuración de todos los módulos, a excepción del de supresión de ruido, se ha mantenido sin cambios con respecto a la antigua topología aplicada.

La nueva topología de grabación estéreo es muy similar a la anterior:

\begin{figure}[H]
	\centering
	\includegraphics[width=1\textwidth]{topologia_stereo_eans2}
	\caption{Nueva topología de grabación estéreo con supresión de ruido.} 
	\label{fig:stereo_eans2}
\end{figure}

La principal diferencia la encontramos en el uso de dos canales en el tipo de dispositivo. Además, la ganancia aplicada es menor, 10\gls{dB}, con el objetivo de no saturar el audio al grabarse desde dos micrófonos.

Tras realizar las modificaciones necesarias, se deben cambiar los ficheros \gls{ACDB} del teléfono por los que tienen la nueva topología.

Una vez aplicados los cambios, se han realiza dos tipos de pruebas:

\begin{enumerate}
	\item Comprobación de la topología aplicada: para todos los casos de uso, en los que se ha cambiado la topología aplicada, se comprueba, si conectando el dispositivo en tiempo real al programa \gls{QACT}, se aplica la topología correspondiente en cada caso. Los resultados han sido todos satisfactorios, aplicándose las nuevas topologías en la grabación mono y estéreo tal y como se han configurado.
	\item Mediante el programa Adobe Audition se comprueban los resultados de estas topologías: con la aplicación Grabadora realizamos una grabación mono antes de aplicar la nueva topología y después de aplicarla. Los resultados de ambos audios analizados con Audition son los siguientes:
	\begin{figure}[H]
		\centering
		\subfigure[Con supresión de ruido activada.]{\includegraphics[width=0.7\textwidth]{audition_eans_activo}}
		\subfigure[Con supresión de ruido desactivada.]{\includegraphics[width=0.7\textwidth]{audition_eans_desactivado}}
		\caption{Comparación de una grabación mono con y sin supresión de ruido.} \label{fig:comparacion_eans_on_off}
	\end{figure}
	
	Como se puede observar, cuando no se aplica la supresión de ruido, el ruido de fondo en todas las frecuencias es mucho más apreciable. Por tanto, se puede concluir que las nuevas topologías cumplen su función de manera satisfactoria.
\end{enumerate}

\subsection{Para el correcto funcionamiento de aplicaciones de llamadas VoIP} \label{sec:llamadas_VoIP}
Las aplicaciones que utilizan llamadas \gls{VoIP} suelen tener comportamientos erráticos. Teóricamente, cuando una llamada \gls{VoIP} se establece, debería interpretarse como una llamada tradicional, al menos a nivel de \gls{ADSP}, aplicando una topología de voz que combine los caminos de recepción y transmisión. Esto nos permitiría utilizar módulos de procesado de la voz específicos para llamadas.

En la práctica esto puede no ocurrir en todas las aplicaciones debido a que el programador de la aplicación no realiza las peticiones necesarias, para interpretarse de ese modo. En varios casos se acaba interpretando como si se tratase de dos caminos separados y sin relación, uno que actúa como si grabase un audio y otro que lo hace como si estuviese en reproducción. Este comportamiento, aunque no es el ideal, es suficiente para que las llamadas se puedan mantener con un nivel y calidad de audio razonables.

El problema surge en dos aplicaciones de mensajería, Skype y Viber. Ambas sufren del problema comentado anteriormente. Como conseguir que se interpreten como llamadas no se puede forzar, sino que se debe realizar una petición por parte de la aplicación, se ha optado por la siguiente solución:
	
\begin{enumerate}
	\item Crear nuevos tipos de dispositivos: se crean nuevos dispositivos en el \gls{ADSP} que se configuran de igual modo a los dispositivos que se estaban aplicando para la reproducción y grabación de audio simultánea. Esto permite configurar las topologías del área \gls{COPP} de ese dispositivo de manera que no afecten al resto de casos de uso. Si se mantuviesen los dispositivos aplicados anteriormente y se quisiesen configurar para las llamadas \gls{VoIP} podría afectar este cambio a otros casos de uso. Se han creado un total de tres dispositivos:
	\begin{itemize}
		\item{Uno que se asociará a la reproducción por el auricular. Se le ha asociado el identificador 171. }
		\item{Otro que se asociará a la reproducción por el altavoz, cuando la función manos libres esté activa. Se le ha asociado el identificador 172}
		\item{Otro que se asociara a la grabación de audio cuando se esté reproduciendo por el auricular. Este dispositivo debe usar ambos micrófonos para la captación. Se le ha asociado el identificador 173.}
		\item{Para el caso de la grabación de audio cuando se esté reproduciendo por el altavoz no será necesario crear un nuevo dispositivo, pues el que se aplica por defecto en las llamadas \gls{VoIP} de Skype y Viber sólo se usa en ese caso. El identificador de este dispositivo es el 117.}
	\end{itemize}
	\item Declarar nuevos SND\_DEVICE: en el código se han declarado tres nuevos dispositivos, dos de salida y uno de entrada, a los que se asocian los identificadores anteriormente descritos. Además, hay que asociar una ruta \textit{hardware}. En los dispositivos de recepción será la misma que tenían los anteriores dispositivos asociados, pero en el de transmisión se utilizará una que proporcione la grabación dual de los micros:
	\begin{figure}[H]
		\centering
		\subfigure[Dispositivos de salida.]{\includegraphics[width=74mm]{snd_voip_id}}
		\subfigure[Dispositivo de entrada.]{\includegraphics[width=73mm]{snd_voip_id_in}}
		\subfigure[Dispositivos de salida.]{\includegraphics[width=74mm]{snd_voip_mixer}}
		\subfigure[Dispositivo de entrada.]{\includegraphics[width=73mm]{snd_voip_mixer_in}}
		\caption{Asociación de rutas del \gls{ADSP} y del códec de audio para los nuevos dispositivos.} \label{fig:snd_voip_asociacion}
	\end{figure}
	\item Seleccionar los nuevos SND\_DEVICE: una vez declarados y asociados los nuevos dispositivos, se debe seleccionar cuando deben utilizarse. En los métodos platform\_get\_output\_snd\_device y platform\_get\_input\_snd\_device se sustituyen los dispositivos seleccionados anteriormente por la selección de los nuevos.
	\item Para el caso de la reproducción por el altavoz en la función manos libres, se debe hacer uso del códec dedicado del altavoz, para ello en el fichero audio\_hw.c indicamos que se rute por ahí:
	\begin{figure}[H]
		\centering
		\includegraphics[width=1\textwidth]{snd_voip_smpa}
		\caption{Rutado del nuevo dispositivo de manos libres por el códec de audio propio del altavoz.} 
		\label{fig:snd_voip_smpa}
	\end{figure}
	\item Hacer la \gls{HAL} compatible a los nuevos dispositivos: se han añadido los nuevos dispositivos a las comparaciones en las que los anteriores dispositivos se incluían.
	\item Compilar la \gls{HAL} y sustituir las librerías generadas en el teléfono. Es necesario estar conectado como superusuario y hacer el sistema compatible a la escritura.
	\item Sustituir los ficheros \gls{ACDB} del teléfono por los que incluyen los nuevos dispositivos. Al reiniciar se tienen aplicados los cambios.
\end{enumerate}

Para probar dichas implementaciones se ha realizado la misma prueba con ambas aplicaciones. Sobre un dispositivo A con la implementación realizada se ha realizado lo siguiente: 

\begin{enumerate}
	\item Conectado al dispositivo A mediante la herramienta \gls{ADB}, desde un dispositivo B se ha realizado una llamada \gls{VoIP} al dispositivo A. Al coger la llamada se comprueban los identificadores \gls{ACDB} que se asocian en el dispositivo A, que son 171 y 173. Se pone el dispositivo A en modo manos libres y se vuelve a comprobar los identificadores, que son el 172 y el 117. Se vuelve a pasar al modo normal y se vuelven a comprobar los identificadores, que vuelven a ser 171 y 173. Se finaliza la llamada. Los resultados han sido los esperados, aplicándose bien los identificadores que relacionan los dispositivos con las rutas del \gls{ADSP}.
	\item Con el dispositivo A conectado en tiempo real con el programa \gls{QACT}, desde un dispositivo B se ha realizado otra llamada \gls{VoIP} al dispositivo A. Al coger la llamada se comprueba si el camino aplicado tanto de recepción como de transmisión, en el dispositivo A, se corresponde con los identificadores \gls{ACDB}. Se pone el dispositivo A en manos libres y se comprueba si los caminos aplicados se corresponden con los identificadores. Los resultados han sido los esperados, seleccionándose correctamente los caminos de recepción y transmisión que se han añadido nuevos en el \gls{ADSP}.
	\item Conectado al dispositivo A mediante la herramienta \gls{ADB}, se realiza una llamada \gls{VoIP} desde un dispositivo B al dispositivo A. Cuando se contesta se pone en modo manos libres y se comprueba mediante el comando tinymix la ganancia de ambos micrófonos. Esta ganancia, al estar utilizando ambos, debería estar puesta al mismo valor. Una captura de los valores durante la prueba se muestra a continuación:
	\begin{figure}[H]
		\centering
		\includegraphics[width=0.4\textwidth]{snd_voip_dmic}
		\caption{Ganancia de los micrófonos en llamadas VoIP.} 
		\label{fig:snd_voip_dmic}
	\end{figure}
	En la imagen anterior se puede comprobar como cuando el dispositivo A está en manos libres tienen el mismo valor, pero al cambiar al modo normal tiene un mayor valor el micro principal, que es el que está habilitado. El resultado, por tanto, es el esperado.
\end{enumerate}